\chapter*{Vorwort}
Schön, dass Du Dich entschieden hast mit dieser Arbeit zu lernen. Vorweg möchte ich raten, nicht nur die Lektüre zu lesen, sondern zum besseren Verständnis, parallel zu jedem abgeschlossenen Kapitel, die jeweiligen Übungsaufgaben zu erledigen. Wenn Du so vorgehst, merkst du schnell, ob Du alles richtig verstanden hast oder Du Dich noch intensiver mit dem Thema befassen musst. Manchmal kann es sich lohnen, nach einem Abschnitt direkt mit den Übungsaufgaben zu beginnen und erst weiter zu lesen, wenn ein neuer Aufgabentyp eines Kapitels gestellt wird.

Außerdem möchte ich hier vorweg darauf hinweisen, dass ich keinerlei Garantien auf Korrektheit oder Vollständigkeit übernehme. Du kannst mir sehr gerne Fehler berichten oder mich auf Unvollständigkeiten hinweisen. Fehler, seien es inhaltliche, grammatikalische oder Rechtschreibfehler, werde ich umgehend korrigieren. Ergänzungen werde ich vornehmen, sofern ich die Zeit dafür erübrigen kann.

Nach Abschluss der Arbeit sind einige TODOs offen geblieben. Im Quelltext sind die entsprechenden Stellen mit "`\%TODO"' markiert. Wenn Du Interesse hast, kannst Du Dich gerne um die TODOs kümmern und mir die Ergänzungen zuschicken.

Dieses Dokument ist auf Grundlage der Vorlesung und dem Skript von Professor Gemmar der Hochschule Trier entstanden. Dieses Dokument ist frei von Rechten Dritter. Alle Abbildungen und Texte wurden von mir, Kajetan Weiß, verfasst. Wenn Du das Dokument für gut genug befindest, würde es mich sehr freuen, wenn Du es anderen zugänglich machen würdest. Zwei Bedingungen zur Verbreitung stelle ich: Erstens, Vorwort und Nachwort müssen erhalten bleiben. Zweitens, wenn Du Änderungen oder Ergänzungen vornimmst, bist Du herzlich dazu eingeladen dies zu tun, stelle bitte an entsprechender Stelle oder am Anfang oder am Ende des Dokuments klar, welche Änderungen oder Ergänzungen Du vorgenommen hast.

Bei Fragen oder Unklarheiten kontaktiere mich bitte per E-Mail oder auf GitHub: \\
\texttt{weissk@hochschule-trier.de} \\
\url{https://github.com/LittleEntity/DigitaleSchaltungen}

Jetzt wünsche ich Dir viel Erfolg beim Lernen, verstehen und lösen der Herausforderungen in der Veranstaltung.

\includegraphics[scale=0.2]{Unterschrift_small.png}\\
\textsl{Kajetan Weiß, Trier den \today}